%%%%%%%%%%% INFORMAÇÕES BÁSICAS %%%%%%%%%%%%%%%%%%%%%
% Versão: 		1.0
% Autor:		Rodrigo Guimarães
% Finalidade:	Descritivo do SPT para o Trabalho 2
%				da disciplina SI 2016.2
%%%%%%%%%%% FIM INFORMAÇÕES BÁSICAS %%%%%%%%%%%%%%%%


%%%%%%%%%%% INCLUSÃO DE PACOTES %%%%%%%%%%%%%%%%
% Classe Base do Documento: Artigo
\documentclass [12pt]{article}
% Layout do Papel
\usepackage{../../Layout/Layout_Documento}
% Case não funcione, verificar o diretório do arquivo 'Layout_Documento.sty'
%\usepackage{lipsum}
%%%%%%%%%%% FIM INCLUSÃO DE PACOTES %%%%%%%%%%%

%%%%%%%%%%% INFORMAÇÕES BÁSICAS %%%%%%%%%%%%%%%
\title {Contrução SI}
%%%%%%%%%%% FIM INFORMAÇÕES BÁSICAS %%%%%%%%%%%

%%%%%%%%%%% DOCUMENTO %%%%%%%%%%%%%%%%%%%%%%%%%
\begin{document}
	\inserirTitulo

	O padrão arquitetural implementado foi o modelo MVC.
	
	\emph{Model-view-controller} (MVC), em português modelo-visão-controlador, é um padrão de arquitetura de software (\emph{design pattern}) que separa a representação da informação da interação do usuário com ele. O modelo consiste nos dados da aplicação, regras de negócios, lógica e funções. Uma visão pode ser qualquer saída de representação dos dados, como uma tabela ou um diagrama. É possível ter várias visões do mesmo dado, como um gráfico de barras para gerenciamento e uma visão tabular para contadores. O controlador faz a mediação da entrada, convertendo-a em comandos para o modelo ou visão. As ideias centrais por trás do MVC são a reusabilidade de código e separação de conceitos.

	Os diagramas de classes ficaram muitos extensos, dessa forma as imagens serão enviadas fora deste arquivo, de modo a possibilitar uma visualização de todos. Vale informar que a classe~\emph{Erro}, faz conexões com todas classes do pacote~\emph{bancoDados.manipuladores} e do~\emph{trabalhoFeliz}, por isso não foi posta, para não dificultar a visualização.

	Os requisitos funcionais estão descritos como:
	
	\begin{enumerate}
		\item \textbf{Cadastrar Empregado:} O sistema cumpriu com o esperado;
		\item \textbf{Analisar Contrato:} O sistema cumpriu em parte, pois todos os funcionários cadastrados foram considerados como empregados;
		\item \textbf{Gerar holerite:} O sistema cumpriu com o esperado;
		\item \textbf{Emitir holerite:} O sistema cumpriu com o esperado;
		\item \textbf{Calcular os bônus:} O sistema cumpriu com o esperado, embora não tenha sido desenvolvido um método para isto;
		\item \textbf{Calcular os ônus:} O sistema cumpriu com o esperado, embora não tenha sido desenvolvido um método para isto;
		\item \textbf{Resgatar Salário Base:} O sistema cumpriu com o esperado;
		\item \textbf{Controlar Férias:} O sistema cumpriu com o esperado;
		\item \textbf{Controlar Parcelas $13$º:} O sistema cumpriu com o esperado;
		\item \textbf{Controlar Faltas:} O sistema cumpriu com o esperado;
		\item \textbf{Controlar alíquota do INSS:} O sistema cumpriu com o esperado;
		\item \textbf{Controlar alíquota do IRRF:} O sistema cumpriu com o esperado;
		\item \textbf{Controlar vale-transporte:} O sistema cumpriu com o esperado;
		\item \textbf{Somar às despesas:} O sistema cumpriu com o esperado.
	\end{enumerate}
	
	Para o conhecimento dos requisitos originais, consultar o arquivo~\emph{\textbf{LevantamentoRequisitos.pdf}}.

\end{document}
%%%%%%%%%%% FIM DOCUMENTO %%%%%%%%%%%%%%%%%%%%%
