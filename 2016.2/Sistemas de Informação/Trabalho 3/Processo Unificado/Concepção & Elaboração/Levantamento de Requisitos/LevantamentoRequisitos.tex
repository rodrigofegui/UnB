%%%%%%%%%%% INFORMAÇÕES BÁSICAS %%%%%%%%%%%%%%%%%%%%%
% Versão: 		1.0
% Autor:			Rodrigo Guimarães
% Finalidade:	Visão da Empresa para o sistema
%%%%%%%%%%% FIM INFORMAÇÕES BÁSICAS %%%%%%%%%%%%%%%%


%%%%%%%%%%% INCLUSÃO DE PACOTES %%%%%%%%%%%%%%%%
% Classe Base do Documento: Artigo
%\documentclass [12pt, landscape]{article}
\documentclass [12pt]{article}
% Layout do Papel
\usepackage{../../Layout/Layout_Documento}
% Case não funcione, verificar o diretório do arquivo 'Layout_Documento.sty'
%\usepackage{lipsum}
%%%%%%%%%%% FIM INCLUSÃO DE PACOTES %%%%%%%%%%%


%%%%%%%%%%% INFORMAÇÕES BÁSICAS %%%%%%%%%%%%%%%
\title {Levantamento de Requisitos}
\date {\today}
%%%%%%%%%%% FIM INFORMAÇÕES BÁSICAS %%%%%%%%%%%

%%%%%%%%%%% DOCUMENTO %%%%%%%%%%%%%%%%%%%%%%%%%
\begin{document}
	\inserirTitulo
	
	Os requisitos funcionais estão descritos como:
	
	\begin{enumerate}
		\item \textbf{Cadastrar Empregado:} O sistema deve registrar um novo empregado no sistema, com seus dados pessoais necessários, expedindo uma notificação ao empregado;
		\item \textbf{Analisar Contrato:} O sistema deve verificar a situação corrente do contrato com o empragado, possibilitando ou não o trabalho do mesmo;
		\item \textbf{Gerar holerite:} O sistema deve gerar a holerite com todos os bônus e ônus do empregado, para um mês específico;
		\item \textbf{Emitir holerite:} O sistema deve emitir a holerite, como forma de comprovação do pagamento realizado ao empregado;
		\item \textbf{Calcular os bônus:} O sistema deve cálcular os bônus disponíveis para o empregado, para um mês específico;
		\item \textbf{Calcular os ônus:} O sistema deve cálcular os ônus disponíveis para o empregado, para um mês específico;
		\item \textbf{Resgatar Salário Base:} O sistema deve ser capaz de obter o valor do salário base do empregado, conforme a sua profissão;
		\item \textbf{Controlar Férias:} O sistema deve ser capaz de gerenciar os pedidos de férias e as concessões possíveis;
		\item \textbf{Controlar Parcelas $13$º:} O sistema deve ser capaz de gerenciar o pagamento das parcelas do $13$º salário;
		\item \textbf{Controlar Faltas:} O sistema deve ser capaz de gerenciar as faltas realizadas pelo empregado;
		\item \textbf{Controlar alíquota do INSS:} O sistema deve ser capaz de determinar a quantia a ser deduzida do salário do empregado para o INSS;
		\item \textbf{Controlar alíquota do IRRF:} O sistema deve ser capaz de determinar a quantia a ser deduzida do salário do empregado para o IRRF;
		\item \textbf{Controlar vale-transporte:} O sistema deve ser capaz de deduzir o vale-transporte, caso seja de desejo do empregado;
		\item \textbf{Somar às despesas:} O sistema deve ser capaz de somar o pagamento realizado aos empregados às despesas da empresa.
	\end{enumerate}
	
%	\begin{LevantarRequisitos}
%		{Cadastrar Empregado}
%		{O sistema deve registrar um novo empregado no sistema, com seus dados pessoais necessários, expedindo uma notificação ao empregado}
%	\end{LevantarRequisitos}
		
%	\begin{LevantarRequisitos}
%		{Analisar Contrato}
%		{O sistema deve verificar a situação corrente do contrato com o empragado, possibilitando ou não o trabalho do mesmo}
%	\end{LevantarRequisitos}
	
%	\begin{LevantarRequisitos}
%		{Gerar holerite}
%		{O sistema deve gerar a holerite com todos os bônus e ônus do empregado, para um mês específico}
%	\end{LevantarRequisitos}
	
%	\begin{LevantarRequisitos}
%		{Emitir holerite}
%		{O sistema deve emitir a holerite, como forma de comprovação do pagamento realizado ao empregado}
%	\end{LevantarRequisitos}
	
%	\begin{LevantarRequisitos}
%		{Calcular os bônus}
%		[x]
%		{O sistema deve cálcular os bônus disponíveis para o empregado, para um mês específico}
%	\end{LevantarRequisitos}
	
%	\begin{LevantarRequisitos}
%		{Calcular os ônus}
%		[x]
%		{O sistema deve cálcular os ônus disponíveis para o empregado, para um mês específico}
%	\end{LevantarRequisitos}
	
\end{document}
%%%%%%%%%%% FIM DOCUMENTO %%%%%%%%%%%%%%%%%%%%%
