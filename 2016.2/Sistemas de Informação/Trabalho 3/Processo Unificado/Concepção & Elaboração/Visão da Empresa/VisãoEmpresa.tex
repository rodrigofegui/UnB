%%%%%%%%%%% INFORMAÇÕES BÁSICAS %%%%%%%%%%%%%%%%%%%%%
% Versão: 		1.0
% Autor:			Rodrigo Guimarães
% Finalidade:	Visão da Empresa para o sistema
%%%%%%%%%%% FIM INFORMAÇÕES BÁSICAS %%%%%%%%%%%%%%%%


%%%%%%%%%%% INCLUSÃO DE PACOTES %%%%%%%%%%%%%%%%
% Classe Base do Documento: Artigo
\documentclass [12pt]{article}
% Layout do Papel
\usepackage{../../Layout/Layout_Documento}
% Case não funcione, verificar o diretório do arquivo 'Layout_Documento.sty'
%\usepackage{lipsum}
%%%%%%%%%%% FIM INCLUSÃO DE PACOTES %%%%%%%%%%%

%%%%%%%%%%% INFORMAÇÕES BÁSICAS %%%%%%%%%%%%%%%
\title {Visão Geral do Sistema}
\date {\today}
%%%%%%%%%%% FIM INFORMAÇÕES BÁSICAS %%%%%%%%%%%

%%%%%%%%%%% DOCUMENTO %%%%%%%%%%%%%%%%%%%%%%%%%
\begin{document}
	\inserirTitulo

	É proposto o desenvolvimento de um sistema de gerenciamento da folha de pagamento dos empregados, que visa informatizar os cálculos sobre os bônus e os ônus para cada mês comercial trabalhado. O objetivo do sistema é recolher os dados sobre o empregado e o seu trabalho e gerar, assim, a holerite corrente conforme esses dados e armazená-la num histórico. O sistema deverá ser capaz de acolher empregados novos na empresa, assim como, para os que permanecerem empregados, calcular, automaticamente, o valor dos pagamentos, conforme as leis vigentes e excentricidades previamente acordadas em contrato. A cada mês trabalhado, o sistema deverá resgatar o salário base do funcionário, realizar os controles: de férias; das parcelas do 13º salário; das faltas; da alíquota do INSS; da alíquota do IRRF; e do vale transporte, emitir a holerite e somar às despesas da empresa o salário pago.
	
	Vale informar que, para a empresa, este gerenciamento sobre a folha de pagamento dos empregados é de grande importância, pois todas os dados e as informações são ou coletados ou gerados automaticamente, com excessão dos dados pessoais do empregado; dessa forma, todo o processo requer menos tempo de execução, se comparado com o trabalho manual, e possibilita a escalabilidade da solução, conforme o avanço da empresa. Isto só vale por se tratar de um projeto realizável, ou seja, a solução é desenvolvida num período aproveitável da empresa, conseguindo atingir a implementação da solução.
	
	Quanto aos desenvolvedores, há plena condição de desenvolvimento de tal projeto, uma vez que o investimento, tanto de capital quanto de tempo, para tal implantação é baixo e há oportunidade para o desenvolvimento, inviabilizando uma compra de uma solução já existente, correndo o risco da mesma não se adequar aos requisitos específicos da empresa.
	
\end{document}
%%%%%%%%%%% FIM DOCUMENTO %%%%%%%%%%%%%%%%%%%%%
