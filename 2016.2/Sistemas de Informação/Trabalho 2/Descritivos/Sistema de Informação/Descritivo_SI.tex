%%%%%%%%%%% INFORMAÇÕES BÁSICAS %%%%%%%%%%%%%%%%%%%%%
% Versão: 		1.0
% Autor:		Rodrigo Guimarães
% Finalidade:	Descritivo do SPT para o Trabalho 2
%				da disciplina SI 2016.2
%%%%%%%%%%% FIM INFORMAÇÕES BÁSICAS %%%%%%%%%%%%%%%%


%%%%%%%%%%% INCLUSÃO DE PACOTES %%%%%%%%%%%%%%%%
% Classe Base do Documento: Artigo
\documentclass [12pt]{article}
% Layout do Papel
\usepackage{../../Layout/Layout_Documento}
% Case não funcione, verificar o diretório do arquivo 'Layout_Documento.sty'
%\usepackage{lipsum}
%%%%%%%%%%% FIM INCLUSÃO DE PACOTES %%%%%%%%%%%

%%%%%%%%%%% INFORMAÇÕES BÁSICAS %%%%%%%%%%%%%%%
\title {Descritivo}
\subtitle {Sistema de Informação}
\date {\today}
%%%%%%%%%%% FIM INFORMAÇÕES BÁSICAS %%%%%%%%%%%

%%%%%%%%%%% DOCUMENTO %%%%%%%%%%%%%%%%%%%%%%%%%
\begin{document}
	\inserirTitulo

	O sistema selecionado para o desenvolvimento do trabalho foi o Sistema de Processamento de Transação (SPT), em especial o Sistema de Folha de Pagamento. Tendo como organização base a descrita no arquivo~\emph{Descritivo\_Organização}. Neste arquivo será descrito, brevemente, os aspectos característico deste SI, além do comportamento de tal SI na organização.
	
	\section{Conceitualização}	
	
	Os SPTs são os sistemas de informação que executam e registram informações transacionais. Essas rotinas são realizadas pelo nível operacional da organização~\cite{Audy:2009}.

	As informações transacionais compreendem todas as informações contidas num único processo de negócios ou unidade de trabalho, e seu propósito principal é apoiar a realização das tarefas operacionais diárias. Além disso, as organizações utilizam as informações transacionais quando realizam tarefas operacionais e decisões repetitivas~\cite{Baltzan:2012}.
	
	É notório que quando uma organização decide utilizar a tecnologia da informação os SPTs são os primeiros a serem informatizados. Devido aos benefícios a toda a organização~\cite{Audy:2009}.
	
	Com a padronização dos dados e dos procedimentos relativos às transações há uma facilidade no desenvolvimento de sistemas baseados em computadores~\cite{Audy:2009}.
	
	\section{Contextualização}
	Com o SI pretende-se desenvolver um gerenciamento da folha de pagamento.
	
	Cada funcionário da empresa preenche uma ficha com suas informações pessoais, como: nome, CPF, data de nascimento, cargo e outras mais; para que haja um controle, primeiramente, da mão-de-obra por parte do~\textit{Setor de Recursos Humanos}.
	
	Com base no cargo, será determinado o salário base, que, por sua vez, influenciará nos valores: do salário bruto do mês; das férias; do $13º$ salário; das faltas; alíquotas do INSS~\footnote{Instituto Nacional do Seguro Social, http://www.previdencia.gov.br/} e do IRRF~\footnote{Imposto de Renda Retido na Fonte, http://idg.receita.fazenda.gov.br/acesso-rapido/tributos/IRRF}; e do vale transporte, se o trabalhador escolher recebê-lo. Tendo as respectivas formas de cálculo:
	
	\begin{itemize}
		\item \textbf{Salário Bruto}: É o valor do salário base;
		\item \textbf{Férias}: Pode ser concedido a cada $12$ meses de trabalho. É o valor do salário do mês anterior, acrescido de $1/3$ do salário base e deduzindo as alíquotas do INSS e do IRRF;
		\item \textbf{$13º$ Salário}: Dividido, obrigatoriamente, em $2$ parcelas: a $1º$ entre $01/02$ a $30/11$ e a $2º$ até $20/12$. É o somatório acumulado de cada $1/12$ do salário líquido do mês anterior. Ao receber a $1º$ parcela, o somatório é reiniciado para a $2º$ parcela. O $13º$ salário total é o somatório das $2$ parcelas;
		\item \textbf{Faltas}: Será considerado que o mês comercial tem $30$ dias, dessa forma, cada dia de falta corresponde a uma diminuição de $1/30$, em relação ao salário bruto;
		\item \textbf{Alíquota do INSS}: Será considera o salário bruto na Tabela~\ref{tab:aliINSSS}. Para os salários acima de R\$$5.189,82$, será deduzido o valor de R\$$570,88$;
			\begin{table}[h]
				\centering
				\caption{Alíquotas do INSS}
				\label{tab:aliINSSS}
				\begin{tabular}{r@{ - }l|c}
					\hline
					\multicolumn{2}{c|}{Faixa Salarial (R\$)} & Alíquota (\%) \\ \hline\hline
						0                 & 1.556,94 & 8  \\
						1.556,95          & 2.594,92 & 9  \\
						2.594,93          & 5.189,82 & 11 \\ \hline
				\end{tabular}
			\end{table}
		\item \textbf{Alíquota do IRRF}: Será considerado o salário após a dedução do INSS na Tabela~\ref{tab:aliIRRF};
			\begin{table}[h]
				\centering
				\caption{Alíquotas do IRRF}
				\label{tab:aliIRRF}
				\begin{tabular}{r@{ - }l|c|c}
					\hline
					\multicolumn{2}{c|}{Faixa Salarial (R\$)} & Alíquota (\%) & Parcela a deduzir (R\$) \\ \hline
					0                   & 1.903,98            & -             & -                       \\
					1.903,99            & 2.826,65            & 7,5           & 142,80                  \\
					2.826,66            & 3.751,05            & 15,0          & 354,80                  \\
					3.751,06            & 4.664,68            & 22,5          & 636,13                  \\
					4.664,68            & INF                 & 27,5          & 869,36                  \\ \hline
				\end{tabular}
			\end{table}
		\item \textbf{Vale Transporte}: Será deduzido $6\%$ do salário bruto.
	\end{itemize}

	O acompanhamento é contínuo ao longo dos meses, para uma folha de pagamento de qualidade.

\bibliographystyle{ieeetr}
\bibliography{Descritivo_SI}

\end{document}
%%%%%%%%%%% FIM DOCUMENTO %%%%%%%%%%%%%%%%%%%%%
